\chapter{Background}

This chapter provides a closer look at the concepts required to understand this work. The following sections firstly discuss background and related work for topics in particle physics, then machine learning and finally reconfigurable hardware research.

\section{Particle Physics}




particle Collisions 
jets
cern
lhc
level x trigger lhc
Physics experiments are crucial
Big data is crucial for Physics


\section{Machine Learning}



training-validation-test dataset
training - 
inference - 
tensor -
machine learning framework
Transformer Neural Networks are great


\section{Reconfigurable Hardware}



c simulation, cosimulation, synthesis

roof ceiling model / pareto front
latency
resource usepackage
throughput
pipelining
HLS
hardware design (parallel vs serial etc)
FPGA are great for NN
FPGA are very hard-coded -> make the code deployable on any platform with optimal settings automatically
HLS is difficult, so coding hardware in Python is desired -> make it easy for engineers and physicists to design systems
Powerful hardware is king
Metaprogramming allows for optimizations and customisability

