\section{Ethical Considerations}\label{ethical}

The purpose of this project is to advance the next-generation particle physics experiments. There are two main aspects that need to be considered - the development of a hardware-mapped transformer neural network architecture and the easy-to-access translation and optimization toolchain for efficiently expressing and networks in common machine learning frameworks.  

The first feature is aimed at a purely civilian, scientific audience, and it is tailored towards particle collision datasets. With that in mind, it is important to mention that, as with most machine learning research, there is potential for a misuse of the acceleration techniques towards a military or malevolent application that could negatively impact the society. However, this also means that there is a low risk for new emerging threats from this particular work; rather the already present ones could become slightly more serious. Fortunately, this should result in existing harm prevention measures staying intact or solely requiring adjustments to their accuracy or speed thresholds.

With the second element's goal of making the creation and deployment of neural networks more accessible, it could be argued that this may in turn increase the number of high energy physics experiments requiring immense energy consumption, like those at LHC \cite{1-cernfacts}, thus negatively affecting the environment. However, this is considered a very low likely cause of action, as the research work of this project is aimed at helping already running experiments and more importantly, the negative environmental implications (for which there are various mitigation strategies \cite{Guida_2016, 2-capeans2017strategies}) are heavily outweighed by potential beneficial technological advancements coming from the scientific discoveries.

Aside from the aforementioned ethical issues, the project is aimed at benefitting the open-source scientific community world-wide. Its outcome could lead to a much more accessible and efficient inference methods that are applicable in many domains outside particle physics.